\documentclass{article}
\usepackage[utf8]{inputenc}
\usepackage{graphicx}
\usepackage{verbatim}
\usepackage{parskip}
\usepackage{subcaption}
\usepackage{listings}
\usepackage[section]{placeins}
\usepackage{color}
\usepackage{hyperref}
\usepackage{amssymb}

%% ------------------------------------------------------------------------------------------ %%
%% ------------------------------------------------------------------------------------------ %%
%% ------------------------------------------------------------------------------------------ %%


% Document Setup

% Turn of Section Numbers
\setcounter{secnumdepth}{0}

% Format Linking on TOC and LOF
\hypersetup{
    colorlinks=true, %replace the ugly red boxes
    linktoc=all, %add links to TOC sections, subsections also linkable
    linkcolor=black, %replace red
    bookmarks=true, %pdf export has a toc too
}


%% ------------------------------------------------------------------------------------------ %%
%% ------------------------------------------------------------------------------------------ %%
%% ------------------------------------------------------------------------------------------ %%


%Title
\title{COMPSCI 2XC3 Final Lab}
\author{Alexander Eckardt, Om Patel, Marwa Khafagy}
\date{April 2023}

%% ------------------------------------------------------------------------------------------ %%
%% ------------------------------------------------------------------------------------------ %%
%% ------------------------------------------------------------------------------------------ %%


%Image
\newcommand{\figureInsetScaled}[3]
{
    \FloatBarrier{}
    \begin{figure}[ht!]
        \centering
        \includegraphics[width=#3\textwidth]{#1}
        \caption{#2}
    \end{figure}
    \FloatBarrier{}
}

%% ------------------------------------------------------------------------------------------ %%


%Image, Default Scaling
\newcommand{\figureInset}[2]
{
    \figureInsetScaled{#1}{#2}{1}
}

%% ------------------------------------------------------------------------------------------ %%
%% ------------------------------------------------------------------------------------------ %%
%% ------------------------------------------------------------------------------------------ %%


\begin{document}

% Title
\maketitle
\newpage

\tableofcontents
\newpage

\listoffigures
\newpage

%% ------------------------------------------------------------------------------------------ %%

\section{Part 1}
\subsection{Shortest Path Approximations}
\subsection{Experiment Suite 1}
\subsection{Mystery Algorithm}

\newpage
\section{Part 2}
\subsection{A$^{*}$ Implementation}
    See (file)'s $a\_star()$ method.
    
\subsection{A$^{*}$ Discussion}
\begin{itemize}
    \item What issues with Dijkstra’s algorithm is A* trying to address?\\
    The biggest problem with Dijkstra’s algorithm is that it considers all possible paths between nodes, which can lead the algorithm down unideal paths. Thus runtime may be wasted on paths unlikely to lead to the shortest path. 
    Thus, when working with large graphs, Dijkstra’s algorithm becomes unnecessarily slow. 
    \item How would you empirically test Dijkstra’s vs A*?\\
          To empirically test Dijkstra’s vs A*, we would compare the runtimes of finding the shortest path of graphs of varying sizes/densiy. 
    \item If you generated an arbitrary heuristic function (similar to randomly generating weights), how would Dijkstra’s algorithm compare to A*?
    \item What applications would you use A* instead of Dijkstra’s?
\end{itemize}


\newpage
\section{Part 3}
\subsection{Experiment Suite 2}
\subsection{Discussion \& Insight}


\newpage
\section{Part 4}
\subsection{UML Discussion \& Implementation}

\newpage
\subsection{Executive Summary}

\newpage
\subsection{Appendix}

\end{document}
