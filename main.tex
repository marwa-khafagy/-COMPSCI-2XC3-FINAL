\documentclass{article}
\usepackage[utf8]{inputenc}
\usepackage{graphicx}
\usepackage{verbatim}
\usepackage{parskip}
\usepackage{subcaption}
\usepackage{listings}
\usepackage[section]{placeins}
\usepackage{color}
\usepackage{hyperref}
\usepackage{amssymb}

%% ------------------------------------------------------------------------------------------ %%
%% ------------------------------------------------------------------------------------------ %%
%% ------------------------------------------------------------------------------------------ %%


% Document Setup

% Turn of Section Numbers
\setcounter{secnumdepth}{0}

% Format Linking on TOC and LOF
\hypersetup{
    colorlinks=true, %replace the ugly red boxes
    linktoc=all, %add links to TOC sections, subsections also linkable
    linkcolor=black, %replace red
    bookmarks=true, %pdf export has a toc too
}


%% ------------------------------------------------------------------------------------------ %%
%% ------------------------------------------------------------------------------------------ %%
%% ------------------------------------------------------------------------------------------ %%


%Title
\title{COMPSCI 2XC3 Final Lab}
\author{Alexander Eckardt, Om Patel, Marwa Khafagy}
\date{April 2023}

%% ------------------------------------------------------------------------------------------ %%
%% ------------------------------------------------------------------------------------------ %%
%% ------------------------------------------------------------------------------------------ %%


%Image
\newcommand{\figureInsetScaled}[3]
{
    \FloatBarrier{}
    \begin{figure}[ht!]
        \centering
        \includegraphics[width=#3\textwidth]{#1}
        \caption{#2}
    \end{figure}
    \FloatBarrier{}
}

%% ------------------------------------------------------------------------------------------ %%


%Image, Default Scaling
\newcommand{\figureInset}[2]
{
    \figureInsetScaled{#1}{#2}{1}
}

%% ------------------------------------------------------------------------------------------ %%


%Image, Default Scaling
\newcommand{\expOneTestOneFigure}[2]
{
    \begin{figure}[ht!]
        \begin{subfigure}[t]{.48\textwidth}
            \centering
            \includegraphics[width=1\textwidth]{#1B.png}
            \caption{Real Total Distance}
        \end{subfigure}\hfill
        \begin{subfigure}[t]{.48\textwidth}
            \centering
            \includegraphics[width=1\textwidth]{#1A.png}
            \caption{Difference in Total Distance}
        \end{subfigure}
    \caption{#2}
    \end{figure}
}


%% ------------------------------------------------------------------------------------------ %%
%% ------------------------------------------------------------------------------------------ %%
%% ------------------------------------------------------------------------------------------ %%


\begin{document}

% Title
\maketitle
\newpage

\tableofcontents
\newpage

\listoffigures
\newpage

%% ------------------------------------------------------------------------------------------ %%

\section{Part 1}

%% ------------------------------------------------------------------------------------------ %%
%% ------------------------------------------------------------------------------------------ %%
%% ------------------------------------------------------------------------------------------ %%

\subsection{Shortest Path Approximations}

We write the shortest path approximation functions by adding a relaxation dictionary, which counts the number of times each node can be relaxed. This means that our approximation functions limit the number of times we can relax a node.

This is useful to see because we can then increase the number of relaxations to see how many times the actual algorithm relaxes each node. If our approximation returns the same result, then the real algorithm relaxed at least one of its nodes, at maximum, the value of $k$ in our approximation.

%% ------------------------------------------------------------------------------------------ %%
%% ------------------------------------------------------------------------------------------ %%
%% ------------------------------------------------------------------------------------------ %%

\subsection{Experiment Suite 1}

We define 4 tests to see the difference between Dijkstra's Algorithm and the Bellman-Ford Algorithm.

%% ------------------------------------------------------------------------------------------ %%

\subsubsection{Test 1}

Our first Test, named $test1$ in our python file, tests the accuracy of the total distance in the all pairs approximation vs real algorithm, based on the number of relaxations.

$test1$ is defined as follows
\begin{itemize}
    \item $nodeCount$, a constant degree for our graph.
    \item $upperWeight$, the maximum weight a node can have.
    \item $relaxationRange$, A range of relaxations to perform.
    \item $realFunc$, either Bellman-Ford or Dijkstra.
    \item $approxFunc$, either the Bellman-Ford or Dijkstra approximation function.
\end{itemize}

It behaves as follows:
\begin{itemize}
    \item We create a random complete graph.
    \item We iterate over a range of relaxation values.
    \item We iterate over each node in the graph.
    \item We keep track of the distance found by running the real algorithm, and the distance found by running the approximation algorithm with $k$ relaxations.
    \item We sum these distances over each node.
    \item We then repeat for each $k$ in our relaxation range, plotting $k$ vs the accuracy of total distances, defined as the absolute difference of $k$ and $k'$, where $k$ is the total distance found in the actual algorithm, and $k'$ is the distance found by the approximation function.
    \item We plot on a separate graphs the actual values for distance we determined.
\end{itemize}

We get two graphs out of $test1$ per iteration, the real total distance of function vs approximation, and the difference between the two.

The implication of $abs(k - k')$ is that if the approximation matches the actual function, we would get a y value of 0.

We run our $test1$ three times.

Our first instance of $test1$ performs the comparison of Dijkstra vs Dijkstra's Approximation on Graphs of Size 50. The weights can be from anywhere between 1 and 100. Our $relaxationRange$ is from 1 to 50.

Next, we compare of BF vs BF's Approximation on Graphs of Size 25. The weights can be from anywhere between 0 and 100. Our $relaxationRange$ is from 1 to 25. We run this on Graphs of Size 25, because BF on Dense graphs of size 50 is extremely slow.

Next, we compare of BF vs BF's Approximation on Graphs of Size 25. The weights can be from anywhere between -100 and 100. Our $relaxationRange$ is from 1 to 25. We run this on Graphs of Size 25, because BF on Dense graphs of size 50 is extremely slow.

\subsubsection{Test 1 - Results}

\FloatBarrier{}
\expOneTestOneFigure{images/part1/exp1_1a}{$test1a$: Dijkstra vs Dijkstra's Approximation Total Distance}
\FloatBarrier{}

In our first result, we see that on a graph with 50 nodes, It only takes 10 approximations for Dijkstra's Approximation to determine the same result as Dijkstra's.

\FloatBarrier{}
\expOneTestOneFigure{images/part1/exp1_1b}{$test1b$: Bellman-Ford vs Bellman-Ford's Approximation}
\FloatBarrier{}

In our second test, we see than it also take

\FloatBarrier{}
\expOneTestOneFigure{images/part1/exp1_1c}{$test1c$: Bellman-Ford vs Bellman-Ford's Approximation w/ Negative Edge Weights}
\FloatBarrier{}

What we can see here is that the difference is immense. On Graphs with negative edge weights, the approximation fairs far worse than if the graph only contains positive edge weights. However, from the graph we can see that the approximation is slowly approaching the real total distance. 
Thus, we can determine that a likely cause of this is that when a graph has negative edge weights, BF relaxes its nodes far more than if it doesn't have any negative weights.

Thus, we can rerun this experiment, but with a larger range of relaxations. Obviously, as we are running on a larger range, we omit some x values. Our relaxation range is now 1 through to 1000, skipping every 10.

\FloatBarrier{}
\expOneTestOneFigure{images/part1/exp1_1d}{Bellman-Ford vs Bellman-Ford's Approximation w/ Negative Edge Weights Extended}
\FloatBarrier{}

The above graph confirms our suspicion -- on graphs with negative edge weights, we need a lot more relaxations in order for BF approximation to get to the same total distance as the regular BF. In other words, BF relaxes it's nodes a lot more frequently with negative edge weights than it does if it doesn't.

%% ------------------------------------------------------------------------------------------ %%
\newpage
\subsubsection{Test 2}

Our second test is named $test4$ in our python file. It tests the accuracy of the approximation functions' total distances from some single starting node. This is effectively a dialed back test1, but this is a more `real' example of how BF and Dijkstra are used.

$test4$ is defined as follows
\begin{itemize}
    \item $nodeCount$, a constant node count for the size of our Graph.
    \item $lowerWeight$, the lower bound for a possible edge weight.
    \item $upperWeight$, the upper bound for a possible edge weight.
    \item $relaxationRange$, the range of relaxations.
    \item $startingNode$, the node at which we start.
    \item $trials$, the number of trials per relaxation amount.
\end{itemize}

As the graphs are randomly generated, we perform a number of trails per relaxation to get a good feel for the distance. As the graph is the same for each run of the algorithms, the approximation will average to the same degree as the actual algorithm. 

A consequence of our randomly generated graphs is that the starting node doesn't really matter, assuming our generation function is correct.

While this may reduce the precision of the test, we increase our overall accuracy for the purpose of the test.

$test4$ behaves as follows:
\begin{itemize}
    \item Iterate over our Relaxation Range.
    \item Initialize our distance variables to 0.
    \item Repeat $x$ times, generate a new graph and run each algorithm on it, using $k$ relaxations for the approximation functions. Our source node for all algorithms is defined.
    \item Average each distance over the $x$ trials.
    \item Plot the absolute difference in total\_distance.
\end{itemize}

Our concrete running of our $test4$ has these parameters.

$test4a$, where we create graphs of size 25, with edge weights between 0 and 100. Our Relaxation Range is between 1 and 50.
We have a constant starting node (which doesn't really matter) as Node=0. We do 100 trials per data point.

$test4b$, where we create graphs of size 25, with edge weights between -100 and 100. Our Relaxation Range is between 1 and 50.
We have a constant starting node (which doesn't really matter) as Node=0. We do 10 trials per data point.

We choose $n$=25, because Bellman-Ford takes lots of time with large $n$.

\subsubsection{Test 2 - Results}

Note that in cases where our edge weights can be negative, we still run Dijkstra's. In this case, we are seeing how accurate the approximation is to the (incorrect) real algorithm. We don't violate some hidden law.

\figureInsetScaled{images/part1/exp1_4a.png}{$test4a$: Difference in Total Difference of the Approximation Algorithms on 20 node Graphs}{0.8}



\figureInsetScaled{images/part1/exp1_4a.png}{$test4b$: Difference in Total Difference of the Approximation Algorithms on 50 node graphs}{0.8}

What we can determine by looking at these results is that.

We have a really large spike on the one trial, but that is most likely a CPU processing one. We don't have spikes like that on the 25 trial run. Strangely, there is an occurrence of a high spike around the same place in the 20 trial one, implying that there may be something about this particular area (when the number of relaxations hits 6) when running Bellman-Ford.

What we see is that as soon as the number of relaxations hits around 10, the approximations match exactly with the real functions.

%% ------------------------------------------------------------------------------------------ %%

\newpage
\subsubsection{Test 3}

Our third test is named $test5$ in our python file. the accuracy of Dijkstra as nodes increase, but we graph it against the multiple approximations with different relaxation values. These values remain constant.

What this gives us is an estimate as to how many relaxations are required by the Real Dijkstra.

We choose Dijkstra here instead of bellman ford, because we only want to use positive edge weights. If we use negative weights, we necessitate the usage of Bellman-Ford, but what we see in test 4 is that the number of relaxations required by Bellman-Ford with negative weights is significantly higher than BF with positive weights. As Dijkstra runs faster than BF in this case, we choose Dijkstra.

$test5$ is defined as follows
\begin{itemize}
 \item $nodeRange$, a range of nodes.
 \item $upperWeight$, an upper range for our edge weights.
 \item $constantRelaxations$, a set of values for the number of relaxations for our approximation function.
 \item $startingNode$,  a starting node.
 \item $trials$, a constant trial count.
\end{itemize}

It behaves as follows:
\begin{itemize}
    \item Generate a complete graph.
    \item Perform the approximation algorithm, for each value in the set of relaxation number. Get the total distance
    \item Get the total dist of the real Dijkstra's algorithm on the same graph.
    \item Average each of these sums over 10 trials.
    \item Do this for each $n$ in our range of nodes.
\end{itemize}

\newpage
\subsubsection{Test 3 - Results}

\figureInsetScaled{images/part1/exp1_5a.png}{$test5a$: Large Range of Test5}{0.5}
\figureInsetScaled{images/part1/exp1_5b.png}{$test5b$: Zoomed In Range of Test5, Small Range}{0.5}

These results are not averaged. $test5a$ seems smooth because we end up skipping 40 values of $k$ at a time. In $test5b$ shows that the variance of total distance is very high, which cause the spikes.

It happens that the total distance of the graph itself here has high variation from data point to data point, because we only generate one random graph per data point. However, it is interesting to see that the peaks of the real total distance co-respond to the peaks of the approximation total distance.

We can see this very well by looking at the 8 relaxation line -- it matches the black line perfectly. Even 6 relaxations matches quite well, even at around $n$=60. Thus, we can say that below $n=60$, Dijkstra does at most 8 relaxations per node.

Thus, we can disregard any peaks because the approximation behaves the same way -- it was just the variance from the data sample. The variance is due to the random nature of the edge weights, we can reduce this by lowering the maximum edge weight, reducing the number of possible total distances.

What we can see is that the rate at which the approximation function's distance seems to increase linearly as number of nodes $n$ increases. The rate of change is dependant on the number of relaxations, $k$.

What we can see is that to get a good approximation for Dijkstra, we only really need to have a relaxation amount $k$ proportional to the size of the graph. Or, in other words, Dijkstra only needs to relax each node a few times.

At a very small graph size, $k=1$ is a relatively good approximation. However, as $n$ increases, this approximation falls off very quickly. Thus, we only need to relax each node once.

At some point $k=2$ is also a very good approximation, and it stays good for a lot longer.

We can repeat this for all $k$. If $k$ is high enough, for small enough graphs, the approximation is quite good, or as good as possible.

Of course, the higher $k$ value we choose, the better the approximation, but what we can tell from this graph is that to get a good approximation, the value of $k$ depends on the value of $n$.

\FloatBarrier{}
\figureInsetScaled{images/part1/exp1_5c.png}{$test5c$: Accuracy of Approximation with large graph at certain $k$ value}{.5}
\FloatBarrier{}

We see this in our $test5c$. Here, we have a very large graph ($n$ around 1000). We have many values of $k$, but to get a good one we don't really need to do a total Dijkstra's algorithm -- we can approximate it very well with $k$=6 really well, and $k=8$ is already good enough. Again, in other words, Dijkstra will approximately use around 10 relaxations on graphs sized around 1000.
%% ------------------------------------------------------------------------------------------ %%

\newpage
\subsubsection{Test 4}

Our fourth test is named $test6$ in our python file. It behaves similarly to test 5, but instead of increasing the node count, we increase the variance of the edge weights. The idea of this test is to see if the range of values the edges can have affects the total distance of the approximation.

We choose Dijkstra here instead of bellman ford, because we only want to use positive edge weights. If we use negative weights, we necessitate the usage of Bellman-Ford, but what we see in test 4 is that the number of relaxations required by Bellman-Ford with negative weights is significantly higher than BF with positive weights. As Dijkstra runs faster than BF in this case, we choose Dijkstra.

$test6$ is defined as follows
\begin{itemize}
 \item $nodeCount$, a constant node count to generate our graphs with.
 \item $upperWeightRange$, an range upper range for our edge weights.
 \item $constantRelaxations$, a set of values for the number of relaxations for our approximation function.
 \item $startingNode$, a starting node.
 \item $trials$, a constant trial count.
\end{itemize}

It behaves as follows:
\begin{itemize}
    \item Generate a complete graph, with weights ranging from 1 to $w$.
    \item Perform the approximation algorithm, for each value in the set of relaxation number. Get the total distance
    \item Get the total dist of the real Dijkstra's algorithm on the same graph.
    \item Average each of these sums over 10 trials.
    \item Do this for each $w$ in our range of upper weights.
\end{itemize}

\newpage
\subsubsection{Test 4 - Results}

\FloatBarrier{}
\figureInsetScaled{images/part1/exp1_6a.png}{$test6a$: Accuracy of Approximation with small graph, Huge in proportion range}{.5}
\figureInsetScaled{images/part1/exp1_6b.png}{$test6b$: Accuracy of Approximation with medium graph, Huge in proportion range}{.5}
\figureInsetScaled{images/part1/exp1_6c.png}{$test6c$: Accuracy of Approximation with large graph, Large range}{.5}
\FloatBarrier{}

What we can see here is again, the total distance is highly dependant on the range of values the edges can be. If the range is larger, then it makes sense that more relaxations would be required-- the chance that an edge is larger than the sum of two other increases with the larger range, meaning there is a higher chance to relax the node. 

However, just because the chance is higher doesn't mean we need the same amount of relaxations. We see that the disparity between the real and the approximate total distance diverges from one another, but it only takes ~6 relaxations for the approximation to match the real one. This means that as the range of weights increase, the chance for a `problem node' to pop up increases. This problem node starts off with a very high distance, but it needs to have many relaxations to get to its real distance. However, because of our approximation, we have limited the number of relaxations, thus we approximate the distance to be very high.

We see this clearly in $test6a$ - Our maximum edge weight range is 1000, and with one relaxation our total distance is around double the real distance -- meaning we have many problem nodes, where they on average have twice their total distance. 

With more relaxations this value decreases, and eventually settles on the real value.

%% ------------------------------------------------------------------------------------------ %%
%% ------------------------------------------------------------------------------------------ %%
%% ------------------------------------------------------------------------------------------ %%

\subsection{Mystery Algorithm - Two Functions}

As stated, when we run a single source algorithm we end up solving for all the pairs that include that source. Thus, to solve the all-pairs problem, we should run it on all of our nodes (a `potential source').

Thus, if correctness is our only concern, the simplest approach would be running Dijkstra on Positive edge weighted graphs and Bellman-Ford on Negative Weighted Graphs,
on each node of the graph we are given.

Thus solving for the time complexities of our two new algorithms will be based on the time complexities of Dijkstra and Bellman-Ford respectively.

As our new algorithms run our source algorithms on every node of the graph, our new time complexities are multiplied by the number of nodes in the graph, $V$.

Thus, our All-Pairs Dijkstra Algorithm, on dense graphs, will have time complexity $\Theta(V^3)$, as we perform a $\Theta(V^2)$ process $V$ times.

Similarly, our All-Pairs Bellman-Ford Algorithm, on dense graphs, will have time complexity $\Theta(V^4)$, as we perform a $\Theta(V^3)$ process $V$ times.

\subsection{Mystery Algorithm - Mystery}

First, we look at the $mystery()$ function to get a better understanding of how it works.

We see that it calls the auxiliary $init\_d()$ function. This, initializes a 2D list, or a table, populating each position as infinity. The dimensions of the table are $n \times n$, where $n$ is the number of nodes found in the graph we are searching.

Then we go across the table and set the value of any connections in the graph to the weight they are connected by. If node $i$ and $j$ are connected, the $i$th column and $j$th row, (and thus vice versa), will be set to be the distance between $i$ and $j$. Like nodes are initialized as a distance of 0.

Once we have our table, we end up looping over each position in the table. In essence, we are looking at every possible pair of nodes $i$ and $j$. We then use this pair with another node $k$, (which we also loop over).

Thus, we are checking on every possible triplet of nodes $i$, $j$, $k$.

we say if the distance from $i$ to $k$ + $k$ to $j$ is less than what we have stored for the distance between $i$ to $j$, then that means that we can reach $j$ from $i$ in less distance than what we had originally. Thus, we update our dictionary.

Because we do this multiple times for each pair $i,j,k$ we'll end up ensuring that we get the minimum distance for $i$ to $j$, $\forall i, j$.

\subsection{Finding the Time Complexity}

What we also see is that we don't ever loop on the order of the edges in the graph. We do check edges, but only on the order of the number of nodes. In other words, the order of our mystery function is invariant on how many edges the graph has.

Thus, to derive the time complexity of our mystery function, we are going to run it on varying sized graphs.

Because we are going to plot our points on a log / log graph, we first decide what our x axis points will be. Here, we go with powers of two, plus some extra to fill out the graph a little more.

Then, we generate an empty weight graph with $n$ nodes and 0 edges (best case), and time it on the mystery function. Next, we generate random nodes to completely fill the graph and re-time the mystery function again.

We repeat this $k$ times, (here $k$ = 5) so that we average the results (CPU usage / other processes may affect time, so we want to reduce this as a variable).

Having both an empty graph and a filled graph gets us both our best and worst cases.

We also plot the line of $x^3$ and $x^2$ on our log/log graph. This is an intuition based on how the code works, but it is only an intuition. They will, however, give us a benchmark to see the complexity of the mystery algorithm.

Now, running our experiment, we get the following results.

\figureInsetScaled{images/part1/mysteryReal.png}{Real Graph of Graph Size vs Time}{0.5}
\figureInsetScaled{images/part1/mysteryLog.png}{Log Graph of Graph Size vs Time}{0.5}

 On a Log/Log graph, the slope of a linear line is the degree of the polynomial, and what we can see is that our function, in both best and worst case, match the slope of $x^3$ as $x$ goes to infinity. In this case, $x$ is our node count, or $V$; meaning that \textbf{we can conclude that our mystery function is on the order of} $\Theta(V^3)$.

We exclude the $x^2$ and $x^3$ lines on the real graph, as it makes the actual results unviewable.

\newpage
\section{Part 2}
\subsection{A$^{*}$ Implementation}

    See (a\_star.py)'s $a\_star()$ method.

\subsubsection{How the algorithm works}
    \begin{itemize}
        \item The algorthm is provided all of the same input as Dijkstra, with an additional input that is referred to as a heuristic function
        \item The heuristic function is a dictionary which takes an edge and returns a number. This number acts as a guide as to not let the algorithm go down "bad" paths. 
        \item With this new heuristic function, the nodes in the queue are now ordered by their distance from the source plus the numerical value the heuristic function returns (for that spesific node). 
        \item However, this heuristic function does not affect the actual cost/distance between any two nodes (i.e. the cost is perserved). Thus, when directly comparing costs to find the shortest path, the heuristic function is not considered.
    \end{itemize}
    
\subsection{A$^{*}$ Discussion}
\begin{itemize}
    \item What issues with Dijkstra’s algorithm is A* trying to address?\\
        The biggest problem with Dijkstra’s algorithm is that it considers all possible paths between nodes, which can lead the algorithm down unideal paths. Thus runtime may be wasted on paths unlikely to lead to the shortest path. 
        Thus, when working with large graphs, Dijkstra’s algorithm becomes unnecessarily slow. 
    \item How would you empirically test Dijkstra’s vs A*?\\
        To empirically test Dijkstra’s vs A*, we would compare the runtimes of finding the shortest path of graphs of varying sizes/densiy. 
    \item If you generated an arbitrary heuristic function (similar to randomly generating weights), how would Dijkstra’s algorithm compare to A*?\\
        In that case, they may perform similarly, or more likely, Dijkstra’s may outperferm A*. This is because this arbitrary heuristic function could lead the algorithm down even more unideal paths. On the other hand, Dijkstra would be more thorough and would consider all possible paths between nodes. A better heuristic function would estimate the distance between nodes and the destination node. 
    \item What applications would you use A* instead of Dijkstra’s?
        In general, given a good heuristic function, A* is more efficient than Dijkstra’s algorithm. Thus, it would make sense to use A* in applications where there exists a heuristic function that can lead the algorithm in the right direction. 
        An example is finding shortest paths in transportation/navigation systems, since a good heuristic function can be the distance between places (nodes) and the destination (destination node). More generally, any other application with pathfinding with some sort distances (like in video games) would perform better with the A* algorithm. 

\end{itemize}


\newpage
\section{Part 3}
Let us now compare the performance of Dijkstra’s Algorithm and the A* Algorithm. 
\subsection{Experiment Suite 2}

We define our tests using the data given to us that describes the london stations (real world data). Using this data, we will be able to see how the two algorithms perform using a good heuristic for A*.\\
\\
The heuristic we use is the real world distance in km between two stations using their latitude and longitude. This is done using the geopy library and the distance function which uses the haversine formula.\\

\subsubsection{Test 1}

In this test, we will compare the performance of the two algorithms on all combinations of stations. This will be a general experiment so we can see how the two algorithms perform in general.\\
\\
The test behaves as follows: \\

\begin{itemize}
    \item We parse through the two files given to us. With the \emph{london\_connections} file, we put the data in the WeightedGraph data structure (which is slightly modified to also store the line between two stations). With the \emph{london\_stations} file, we put the data in a dictionary where the key is the station name and the value is another dictionary with the latitude and longitude, name, zone, total lines, and rail of the station. \\
    \item Then, for each combination of stations, we run the two algorithms, record the time it takes to run then add those times to their own total time variables.
    \item Finally, we divide the total time by the number of combinations to get the average time for each algorithm.
    \item We then repeat this for 15 trials and plot the results on a graph.
    
\end{itemize}

We only conduct 15 trials as finding the shortest path between all combinations of stations is a very time consuming process and only 15 trials are realistically needed to get a good idea of how the algorithms perform.\\
\newpage
\subsubsection{Test 1 Results}

\figureInsetScaled{images/part3/General.png}{General Performance A* and Dijkstra’s Algorithm}{1}

From the graph above, we can see that although both algorithms have the same spikes and dips (these may be due to a number of reasons such as the computer being busy with other processes), A* is generally twice as fast as Dijkstra’s Algorithm. This is expected as the A* algorithm uses a heuristic function to guide the algorithm in the right direction and Dijkstra's algorithm may waste time on paths that are unlikely to lead to the shortest path.\\
\\
The heuristic function we chose is the real world distance in km between two stations using their latitude and longitude. This proves to be a good heuristic function as a station that has a shorter time between it and the destination station may be leading away from the destination station. Thus, the A* algorithm is able to find the shortest path faster than Dijkstra’s algorithm.\\

\subsubsection{Test 2}

In this test, we will only compare the performance of the two algorithms on the shortest path between stations that are in the same zone. This data is provided to us in the \emph{london\_stations} file. We will conduct this test to see if the algorithm perform differently if the stations are closer together (i.e. in the same zone).\\
\\
The test behaves as follows (similar to Test 1): \\
\\
\begin{itemize}
    \item We parse through the two files given to us. With the \emph{london\_connections} file, we put the data in the WeightedGraph data structure (which is slightly modified to also store the line between two stations). With the \emph{london\_stations} file, we put the data in a dictionary where the key is the station name and the value is another dictionary with the latitude and longitude, name, zone, total lines, and rail of the station. \\
    \item Then, for each combination of stations that are in the same zone, we run the two algorithms, record the time it takes to run then add those times to their own total time variables.
    \item Finally, we divide the total time by the number of combinations to get the average time for each algorithm.
    \item We then repeat this for 15 trials and plot the results on a graph.
    
\end{itemize}

\subsubsection{Test 2 Results}

\figureInsetScaled{images/part3/SameZone.png}{Performance A* and Dijkstra’s Algorithm on Stations in the Same Zone}{1}

As we can see from the above graph that even when finding the shortest path between stations in the same zone, the A* algorithm is still faster than Dijkstra’s algorithm. This is still unsurprising as there would be no reason that the heuristic function would perform any worse in this case.\\
\\
However, what is interesting is that the the different in performance between the two algorithms in this case is much smaller. The spikes and dips remain the same, but the A* algorithm is only somewhat marginally better than Dijkstra’s algorithm. This is likely due to the fact that the stations are closer together and thus the possibility that Dijkstra’s will traverse down the wrong path is less. \\

\subsection{Discussion \& Insight}

From the tests we ran, we can generally conclude that the A* algorithm is faster than Dijkstra’s algorithm. The use of the heuristic by A* is hugely beneficial as it allows the algorithm to traverse down the right path and thus find the shortest path faster. However, when it comes to certain combinations of station (like the stations being in the same zone), the difference in performance is not as significant. \\
\\
From the results we gathered when the stations of the shortest path were in the same zone, we can theorize that when the stations are closer together, the heuristic function is not as useful as it is when the stations are farther apart.\\

\newpage
\section{Part 4}
\subsection{UML Discussion \& Implementation}

First, we port over everything from the other parts into our Part 4. Then, we construct the abstract classes as described by the UML.

Now that we have this, we modify the parts of our functions which access parts of our graph.

The only real function we need to switch, which complies with the Encapsulation design principle, is switching the lines

\begin{verbatim}
    for neighbour in G.adj[current_node]:
\end{verbatim}

with the code

\begin{verbatim}
    for neighbour in G.get_adj_nodes(current_node):
\end{verbatim}

We do this because we don't want to expose how the Graph stores adjacent nodes. If we were to create a new Graph class that doesn't contain an adjacency list, we'd raise an error.

This is the only thing we can modify in our algorithms because the other methods defined in the UML diagram for the graph class are already used.

However, a clear problem (in the exposing sense), is getting the nodes of the graph. We are not given a $get\_nodes()$ function, meaning, in all of our algorithms, we are still accessing the adjacency list to get the nodes (by getting the key list). This would need to be added to the UML Diagram.

We are however, given a $get\_nodes\_count()$ function. While this naively may solve the solution now, it isn't how we should actually solve the problem. This is discussed further below.


% Refactor you code to adhere to this diagram. Furthermore, consider the points listed below and discuss
% these points in a section labelled Part 4 in your report (where appropriate). Python is a different from
% Java when it comes to things like interfaces, public/private, etc. Do some research regarding how these
% things are done (if at all) in Python.
After conducting some research, it became evident to us that python doesn't traditionally support Interfaces. 

To get around this roadblock, python users instead implement the interfaces as abstract classes.
This is done using the ABC and abstract class methods from the abc python library. 

% • The A_Star class Figure 2 does not exactly line up with the one I asked you to implement in Part
% 2. Do no rewrite you’re A* algorithm! Instead, treat the A_Star class from Figure 2 as an
% “adapter”.
The reason the A\_star class does not exactly line up with the one from part two is because the function in part two took in an additional input that was the heuristic funciton (implemented as a dictionary). 

To get around this while complying with the open close principal we used the original function from part two as an auxiliary funciton that the function $calc\_sp()$ (given in the UML) uses. This auxiliary function remains unchanged, except for a rename, as we now call it $calc\_sp\_aux()$. 

On the other hand, the $calc\_sp()$ acts as an adapter, as it returns $calc\_sp\_aux()$, passing the parameters it was given as well as the heuristic function it creates.

For robustness, we first check to make sure that the graph being used is of type HeuristicGraph, to ensure that there does indeed exits a heuristic function. 
In the case that the graph given as input is not a HeuristicGraph, a default heuristic function that always returns 0 is used. This would effectively turn A* back into Dijkstra. 

% • Discuss what design principles and patterns are being used in Figure 2.
The design pattern that is being used in the given UML diagram is the Strategy pattern. This is seen in the relationship between the A\_star, Bellman\_Ford, Dijkstra, and SPAlgorithm classes. 

This complies with the open-close principle. This is because by doing so, 
we can very easily add new algorithms in the future while not changing any of the old code. In addition to this, we have designed for interfaces. Although technically 
we implemented abstract classes instead of interfaces, we have still implemented this design principle, since most of our classes are not concrete, and thus
our application can better anticipate for change. 

% • Figure 2 is a step in the right direction, but there is still a lot of work that can be done. For
% example, how graph Nodes are represented is being exposed to the rest of the application. That
% is, the rest of the application assume nodes are represented by integers. What if this needs to
% be changed in the future? For example, what if later on we need to switch nodes to be
% represented at Strings? Or what if nodes should carry more information than their names?
% Explain how you would change the design in Figure 2 to be robust to these potential changes.
We would have to create new functions that encapsulate the way nodes are represented. Right now, we treat the values of the nodes as node identifiers. We should separate these parts of the node to make the functionality more robust. This is why we can't and shouldn't access the nodes by iteration over the number of them -- we make a major assumption that the node values are the keys, and that is something we really shouldn't be assuming.

To make the design more robust, we should make a separate Node abstract class
and have the nodes in the graph be of type Node instead. Within the Node class, we would use generics to have a variable the node value. This way, the node can have any type. In the case where the nodes store more than one value all that would need to be done is extend the node class with a new class that contains the other values that must also be stored. 

Instead of getting the nodes by parsing the graph's adjacency list, we should maybe make a function like \verb|get_node_ids()|, which returns an enumerable object containing all node id's. Then we can have a \verb|get_node(id)| function which then returns the node, which could then point us to the value we are storing.

Then, the graph could contain multiple types of nodes (although it shouldn't), but we have the robustness to do so.

% • Discuss what other types of graphs we could have implement “Graph”. What other
% implementations exist? ***(TODO)**
Currently, the only graph we have is a Heuristic Graph, and an directed weighted graph.

For more traditional graphs, we could also implement undirected weighted graphs, which adds two directed edges whenever we push a new edge.

We could also have a proxy class which could act as an Unweighted Graph. It would effectively act as a weighted graph, but we ensure that all weights are the same. This may cause some complications when we look at the UML diagram, but these could be rectified with some refactoring.

\newpage
\section{Executive Summary}

\subsection{Part 1}

\subsection{Part 2}

\subsection{Part 3}

A* gerenally performs much better than Dijkstra’s algorithm, as it performs approximately twice as fast as Dijkstra’s algorithm. This is because A* uses a heuristic function to compute the real timedistance between the current node and the goal node. This allows A* to skip over nodes that are not on the shortest path, and thus reducing the number of nodes that need to be visited.\\
\\
However, when stations are closer together (which can be determined from their zone), Dijkstra’s performs better and the different in performance between the two algorithms is not as significant. This is because the heuristic function is not as accurate when the stations are closer together, and thus A* will not be able to skip over as many nodes.\\

\subsection{Part 4}


\newpage
\section{Appendix}

\subsection{Part 1}

The Approximation functions can be found in the file

\verb|./part1/shortest_path_approx.py|.
The functions are labeled $dijkstra\_approx()$ and $bellman\_ford\_approx()$.

The Experiment Suite Test are found in \verb|./part1/exp1.py|. Our three test functions are called $test1()$, $test4()$ and  $test5()$. The actual tests are ran at the bottom of the file, under the \verb|if __name__ == "__main__":| condition.

The Mystery Function Tests are found in \verb|./part1/mystery_algorithm.py|. There are two functions, $testing()$ (which is were we generate small graphs and run the mystery function on it), and $determiningTimeComplexity()$, which is the function that actually creates the graphs found in Part 1. The actual functions are ran at the bottom of the file, under the \verb|if __name__ == "__main__":| condition.

\subsection{Part 2}

Find the A$^{*}$ algorithm at \verb|./part2/a_star.py|. The function is labeled as $a\_star()$.

\subsection{Part 3}

The experiments ran can be found inside of \emph{.\_/part3\_/part3experiments.py}. The tests used are named \emph{general\_experiments} and \emph{same\_zone\_experiments} \\
\\
The functions used are \emph{a\_star.py} and \emph{Dijkstra.py}. The data structure used can be found in the \emph{WeightGraph.py} file. The actual experiments are ran at the bottom of the file, under the \verb|if __name__ == "__main__":| condition.

\subsection{Part 4}


Find all classes co-responding to the UML Diagram at at \verb|./part4/|. Each class is in it's own designated \verb|.py| file.


\end{document}

\end{document}